\documentclass[12pt,letterpaper]{article}
\usepackage{natbib}

%Packages
\usepackage{pdflscape}
\usepackage{fixltx2e}
\usepackage{textcomp}
\usepackage{fullpage}
\usepackage{float}
\usepackage{latexsym}
\usepackage{url}
\usepackage{epsfig}
\usepackage{graphicx}
\usepackage{amssymb}
\usepackage{amsmath}
\usepackage{mathtools}
\usepackage{bm}
\usepackage{array}
\usepackage[version=3]{mhchem}
\usepackage{ifthen}
\usepackage{caption}
\usepackage{hyperref}
\usepackage{amsthm}
\usepackage{amstext}
\usepackage{enumerate}
\usepackage[osf]{mathpazo}
\usepackage{dcolumn}
\usepackage{lineno}
\usepackage{dcolumn}
\usepackage{mathtools}

\DeclarePairedDelimiter\abs{\lvert}{\rvert}%
\DeclarePairedDelimiter\norm{\lVert}{\rVert}%
\newcolumntype{d}[1]{D{.}{.}{#1}}

\pagenumbering{arabic}


%Pagination style and stuff
\linespread{2}
\raggedright
\setlength{\parindent}{0.5in}
\setcounter{secnumdepth}{0} 
\renewcommand{\section}[1]{%
\bigskip
\begin{center}
\begin{Large}
\normalfont\scshape #1
\medskip
\end{Large}
\end{center}}
\renewcommand{\subsection}[1]{%
\bigskip
\begin{center}
\begin{large}
\normalfont\itshape #1
\end{large}
\end{center}}
\renewcommand{\subsubsection}[1]{%
\vspace{2ex}
\noindent
\textit{#1.}---}
\renewcommand{\tableofcontents}{}
%\bibpunct{(}{)}{;}{a}{}{,}

%---------------------------------------------
%
%       START
%
%---------------------------------------------

\begin{document}

%Running head
\begin{flushright}
Version dated: \today
\end{flushright}
\bigskip
\noindent RH: Characters correlation

\bigskip
\medskip
\begin{center}

\noindent{\Large \bf Effect of coding correlated discrete character correlation on tree inference}
\bigskip

\noindent {\normalsize \sc Thomas Guillerme$^1$$^*$, and Martin D. Brazeau$^1$}\\
\noindent {\small \it 
$^1$Imperial College London, Silwood Park Campus, Department of Life Sciences, Buckhurst Road, Ascot SL5 7PY, United Kingdom.\\}
\end{center}
\medskip
\noindent{*\bf Corresponding author.} \textit{t.guillerme@imperial.ac.uk}\\ 
\vspace{1in}

%Line numbering
\modulolinenumbers[1]
\linenumbers

%---------------------------------------------
%
%       ABSTRACT
%
%---------------------------------------------

\newpage
\begin{abstract}
blablabla
\end{abstract}

\noindent (Keywords: )\\

\vspace{1.5in}

\newpage 


%---------------------------------------------
% LaTeX tips for modifying/editing the document:
%---------------------------------------------
% - You can comment using the percentage sign. I suggest you use the % sign alone for commenting out sections of the text:
%       e.g. "This is a really long sentence %because this sentence is very long." Here the % is used for ignoring the end of the sentence (but for some reason you want to keep track of it).%
%       For comments as in verbose comments, I suggest you use "%MB:":
%       e.g. "This is a really long sentence %because this sentence is very long. %MB: yeah, no shit!" 
% - For optimal version control, write only one sentence per line (for more precise track changes)
% - To build the pdf, use command+B in Sublime.
% - Because of the bibliography, the pdf needs to be build in the same folder that contains "References.bib" and "sysbio.bst"
% - For citing papers, you must put their bibtex reference in the "References.bib" file and then you can use the following sysbio tags:
%        \cite{bibtexBob2000} for citing within a sentence: "Bob (2000)"
%        \citep{bibtexBob2000} for citing within brackets: "(Bob, 2000)"
%        \citep[Before:][-After]{bibtexBob2000} for citing within brackets with additional text: "(Before: Bob, 2000 -After)"
%        \citealt{bibtexBob2000} for citing without brackets: "Bob, 2000"
%        You can put more cites in each \cite tag by separating them with commas.
% - For equation, find every details here: https://en.wikibooks.org/wiki/LaTeX/Mathematics
% - For titles and stuff, the hierarchy goes \section{}, \subsection{}, \subsubsection{} and so forth...
% - For bullet points or enumerations you can use:
%       \begin{itemize}
%           \item my first bullet point/enumeration
%       \end{itemize}
%       With replacing "itemize" by "enumerate" for enumeration.




%---------------------------------------------
%
%       INTRODUCTION
%
%---------------------------------------------
\section{Introduction}

Discrete characters correlation have been studied in the context of Phylogenetic Comparative Methods to test whether two or more characters could have coevolved (Maddison 1990, Evolution; Pagel 1994, Proc B).
However, nobody (that we know of) has looked at what is the effect of actually coding characters that coevolved in a phylogenetic matrix.



-Morphological characters are more and more used: phenomics, tip-dating, Total-Evidence.

-More accurate methods (parsimony vs Bayesian) and better models (Mk, Mk variable from Wright).

-But question remains on which characters to code!

-Don't code correlated characters but what is the actual effect of that? And how many characters are ``secretly'' correlated?

\cite{Davalos01072014}: character correlation + signal in bats
\cite{Grabowski2016}: amount of data needed for testing correlations? CHECK
\cite{Lande1983}: measure correlation CHECK
\cite{Pagel1994}: measure correlation
\cite{Pagel2006}: measure correlation Bayesian
\cite{Maddison1990}: measure binary state character correlation for testing coevol

\section{Methods}

We define characters as being correlated if they give the same phylogenetic information.

\subsection{Measuring correlation between characters}

In order to measure, the difference between two characters, we propose a new distance metric defined as follows:

\subsubsection{Characters difference (CD)}
\begin{equation}
    CD_{(x,y)} = 1 - \left(\frac{\abs*{\frac{\abs*{\sum_{i}^{n}(x_{i} - y_{i})}}{n}-\frac{1}{2}}}{\frac{1}{2}}\right)
\end{equation}

\noindent Where $n$ is the number of taxa with comparable data and $x_i$ and $y_i$ are each characters states for the characters $x$ and $y$ and the taxa $i$ \citep[i.e. the Gower distance;][]{GowerDist}.
Since we are considering differences as being only Fitch-like, we ignore the actual distance between two different character states tokens: two same character states tokens have a difference of $0$ and two different, a difference of $1$ (e.g. $0 - 0 = 0$ or $1 - 8 = 1$).
In order to facilitate and standardise characters states comparison, we standardised each characters by arbitrarily modifying their character states tokens by order of appearance.
In other words, we replaced all the occurences of the first token to be $0$, the second to be $1$, etc.
This way, a character \texttt{A = {2,2,3,0,0,3}} would be standardised as \texttt{A' = {0,0,1,2,2,1}}. Note that in terms of phylogenetic signal, both \texttt{A} and \texttt{A'} are exactly identical (forming three distinct groups).
When the character difference is equal to $0$, it means that characters convey the same phylogenetic signal.
When the character difference is equal to 1 it means it conveys the most different signal.
For example with three characters \texttt{A = {0,1,1,1}}, \texttt{B = {1,0,0,0}} and \texttt{C = {0,1,2,3}}, $CD_{(A,B)} = 0$ and $CD_{(A,C)} = 1$.

This metric respects triangular equality [CHECK THE PROPER NAME] (i.e. $CD_{(A,B)} = CD_{(B,A)}$ and $CD_{(A,C)} = CD_{(B,C)}$).



\subsection{Simulating discrete morphological matrices}

\subsection{Modifying the matrices}

\subsection{Inferring the phylogenies}

\subsection{Comparing the topologies}

\cite{Guillerme2016146}

\subsection{Empirical morphological matrices}


\section{Acknowledgements}
European Research Council under the European Union’s Seventh Framework Programme (FP/2007–2013)/ERC Grant Agreement number 311092.

% Suggested reviewers:
%- Liliana Davalos

\bibliographystyle{sysbio}
\bibliography{References}

\end{document}

