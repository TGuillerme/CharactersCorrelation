\documentclass[12pt,letterpaper]{article}
\usepackage{natbib}

%Packages
\usepackage{fixltx2e}
\usepackage{textcomp}
\usepackage{fullpage}
\usepackage{float}
\usepackage{latexsym}
\usepackage{url}
\usepackage{epsfig}
\usepackage{graphicx}
\usepackage{amssymb}
\usepackage{amsmath}
\usepackage{mathtools}
\usepackage{bm}
\usepackage{array}
\usepackage[version=3]{mhchem}
\usepackage{ifthen}
\usepackage{caption}
\usepackage{hyperref}
\usepackage{amsthm}
\usepackage{amstext}
\usepackage{enumerate}
\usepackage[osf]{mathpazo}
\usepackage{dcolumn}
\usepackage{lineno}
\usepackage{pdflscape}

\usepackage{color,soul}

\DeclarePairedDelimiter\abs{\lvert}{\rvert}%
\DeclarePairedDelimiter\norm{\lVert}{\rVert}%
\newcolumntype{d}[1]{D{.}{.}{#1}}

\pagenumbering{arabic}


%Pagination style and stuff
\linespread{2}
\raggedright
\setlength{\parindent}{0.5in}
\setcounter{secnumdepth}{0} 
\renewcommand{\section}[1]{%
\bigskip
\begin{center}
\begin{Large}
\normalfont\scshape #1
\medskip
\end{Large}
\end{center}}
\renewcommand{\subsection}[1]{%
\bigskip
\begin{center}
\begin{large}
\normalfont\itshape #1
\end{large}
\end{center}}
\renewcommand{\subsubsection}[1]{%
\vspace{2ex}
\noindent
\textit{#1.}---}
\renewcommand{\tableofcontents}{}
%\bibpunct{(}{)}{;}{a}{}{,}

%---------------------------------------------
%
%       START
%
%---------------------------------------------

\begin{document}
%Running head
\begin{flushright}
Version dated: \today
\end{flushright}

\bigskip
\noindent RH: Characters correlation
\bigskip
\medskip
\begin{center}
\noindent{\Large \bf Influence of different modes of morphological character correlation on phylogenetic tree inference}
\bigskip

\noindent{\Large \bf Supplementary material 1 - Character difference}


\noindent {\normalsize \sc Thomas Guillerme$^{1,2,*}$, Abigail I. Pastore$^{1}$, and Martin D. Brazeau$^{2,3}$}\\
% Abigail I. Pastore$^{1}$
\noindent {\small \it 
$^1$School of Biological Sciences, University of Queensland, St. Lucia, Queensland, Australia.\\
$^2$Imperial College London, Silwood Park Campus, Department of Life Sciences, Buckhurst Road, Ascot SL5 7PY, United Kingdom.\\
$^3$Department of Earth Sciences, Natural History Museum, Cromwell Road, London, SW75BD, United Kingdom.\\}

\end{center}
\medskip
\noindent{*\bf Corresponding author.} \textit{guillert@tcd.ie}\\ 
\vspace{1in}

%Line numbering
% \modulolinenumbers[1]
% \linenumbers
\newpage

\section{Definition}

$CD$, the Character Difference metric, is a distance that measures the difference between two characters in terms of shared phylogenetic splits.

For two characters $x$ and $y$, $CD=0$ implies that the characters are identical, i.e. they convey the same phylogenetic signal in term of splits; $CD=1$ implies that the characters are the most different, i.e. they convey the most different phylogenetic signal in terms of splits.

\begin{equation}
    CD_{(x,y)} = \frac{\sum_{i}^{n}\abs{x_{i} - y_{i}}}{n-1}
\end{equation}

\noindent Where $n$ is the number of taxa with comparable data and $x_i$ and $y_i$ are each transformed characters states for the characters $x$ and $y$ and the taxa $i$.
The characters are transformed following \cite{felsenstein2004inferring}'s \textit{xyz} notation where characters states are reorder so that:

\begin{equation}
x'_1 = 1 \text{ and } y'_1 = 1
\end{equation}

\noindent the following character states are then either translated using the same notation for each $i^{th}$ taxa where the state integer is incremented by $+1$ for each new encountered state:

\begin{equation}
\text{If } x_i = x_1 \text{ then } x'_i = 1 \text{; else } x'_i = x'_1 + 1
\end{equation}

\begin{equation}
\text{If } y_i = y_1 \text{ then } y'_i = 1 \text{; else } y'_i = y'_1 + 1
\end{equation}

\noindent For example if:

\texttt{x = {0,0,2,2,1,1}} and \texttt{y = {8,8,0,0,3,3}}

\noindent then:

\texttt{x' = {1,1,2,2,3,3}} and \texttt{y' = {1,1,2,2,3,3}}


\noindent The following proofs consider that the \textit{xyz} notation was applied.

\subsection{Justification for the character states translation}
Because the character states tokens are not representing any meaningful numerical value if the characters are unordered (e.g. there is no assumption that state 0 should come before state 1), they need to be reordered in a standard way so that the first token in the character is renamed to ``1'', the second to ``2'' etc.
For example, in a 3 state character for four taxa: \texttt{X = {1, 0, 2, 1}}, the character tokens that are integers representing the states could equally have been \texttt{X = {E, A, I, E}} or \texttt{X = {blue, red, absent, blue}}.
Therefore, to compare characters fairly, the tokens needs to be translated in a standard way such as the first token is replaced by the symbol ``1'', etc.
In this example \texttt{X = {1, 0, 2, 1}} becomes \texttt{X' = {1, 2, 3, 1}}.

Note that the first two translated tokens are always equal (\texttt{x = {1, ...}} and \texttt{y = {1, ...}}) hence the denominator of $CD$ being $n - 1$ to reflect this equality.
Because of this, $CD$ can only be applied for $n \in [2;\infty)$ (i.e. if $n = 1$, the denominator is 0).

\subsection{Interpretation}
For a character \texttt{X = {1, 0, 2, 1, 0}} implying two splits between five taxa and two characters \texttt{Y = {1, 1, 1, 2, 2}} and \texttt{Z = {0, 2, 1, 0, 2}} implying respectively one and two splits between five taxa, $CD_{(X,Y)} = 0.75$ and $CD_{(X,Z)} = 0$.
This can be interpreted as the probability of characters \texttt{X}, \texttt{Y} and \texttt{X}, \texttt{Z} giving a conflicting phylogenetic signal between the four taxa.
Here \texttt{X}, \texttt{Y} will conflict for 75\% of the taxa and \texttt{X}, \texttt{X} for 0\% of them.

\bibliographystyle{sysbio}
\bibliography{References}

\end{document}



