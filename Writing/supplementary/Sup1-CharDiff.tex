\documentclass[12pt,letterpaper]{article}
\usepackage{natbib}

%Packages
\usepackage{pdflscape}
\usepackage{fixltx2e}
\usepackage{textcomp}
\usepackage{fullpage}
\usepackage{float}
\usepackage{latexsym}
\usepackage{url}
\usepackage{epsfig}
\usepackage{graphicx}
\usepackage{amssymb}
\usepackage{amsmath}
\usepackage{mathtools}
\usepackage{bm}
\usepackage{array}
\usepackage[version=3]{mhchem}
\usepackage{ifthen}
\usepackage{caption}
\usepackage{hyperref}
\usepackage{amsthm}
\usepackage{amstext}
\usepackage{enumerate}
\usepackage[osf]{mathpazo}
\usepackage{dcolumn}
\usepackage{lineno}
\usepackage{dcolumn}
\usepackage{mathtools}

\DeclarePairedDelimiter\abs{\lvert}{\rvert}%
\DeclarePairedDelimiter\norm{\lVert}{\rVert}%
\newcolumntype{d}[1]{D{.}{.}{#1}}

\pagenumbering{arabic}


%Pagination style and stuff
\linespread{2}
\raggedright
\setlength{\parindent}{0.5in}
\setcounter{secnumdepth}{0} 
\renewcommand{\section}[1]{%
\bigskip
\begin{center}
\begin{Large}
\normalfont\scshape #1
\medskip
\end{Large}
\end{center}}
\renewcommand{\subsection}[1]{%
\bigskip
\begin{center}
\begin{large}
\normalfont\itshape #1
\end{large}
\end{center}}
\renewcommand{\subsection}[1]{%
\vspace{2ex}
\noindent
\textit{#1.}---}
\renewcommand{\tableofcontents}{}
%\bibpunct{(}{)}{;}{a}{}{,}

%---------------------------------------------
%
%       START
%
%---------------------------------------------

\begin{document}

%Running head
\begin{flushright}
Version dated: \today
\end{flushright}
\bigskip
\noindent RH: Characters correlation

\bigskip
\medskip
\begin{center}

\noindent{\Large \bf Influence of different modes of morphological character correlation on phylogenetic tree inference}

\bigskip

\noindent{\Large \bf Supplementary material 1 - Character difference}

\bigskip

\noindent {\normalsize \sc Thomas Guillerme$^{1,2,*}$, and Martin D. Brazeau$^{2,3}$}\\
\noindent {\small \it 
$^1$School of Biological Sciences, University of Queensland, St. Lucia, Queensland, Australia.\\
$^2$Imperial College London, Silwood Park Campus, Department of Life Sciences, Buckhurst Road, Ascot SL5 7PY, United Kingdom.\\
$^3$Department of Earth Sciences, Natural History Museum, Cromwell Road, London, SW75BD, United Kingdom.\\}

\end{center}
\medskip
\noindent{*\bf Corresponding author.} \textit{guillert@tcd.ie}\\ 
\vspace{1in}

%Line numbering
% \modulolinenumbers[1]
% \linenumbers
\newpage

\section{Definition}

$CD$, the Character Difference metric, is a distance that measures the difference between two characters in terms of shared phylogenetic splits.
For two characters $x$ and $y$, $CD=0$ implies that the characters are identical, i.e. they convey the same phylogenetic signal in term of splits; $CD=1$ implies that the characters are the most different, i.e. they convey the most different phylogenetic signal in terms of splits.


\begin{equation}
    CD_{(x,y)} = 1 - 2\left(\abs*{\frac{\sum_{i}^{n}\abs{x_{i} - y_{i}}}{n}-\frac{1}{2}}\right)
\end{equation}

\noindent Where $n$ is the number of taxa with comparable data and $x_i$ and $y_i$ are each transformed characters states for the characters $x$ and $y$ and the taxa $i$.
The characters are transformed following \cite{felsenstein2004inferring}'s \textit{xyz} notation where characters states are reorder so that:

\begin{equation}
x'_1 = 1 \text{ and } y'_1 = 1
\end{equation}

\noindent the following character states are then either translated using the same notation for each $i^{th}$ taxa where the state integer is incremented by $+1$ for each new encountered state:

\begin{equation}
\text{If } x_i = x_1 \text{ then } x'_i = 1 \text{; else } x'_i = x'_1 + 1
\end{equation}

\begin{equation}
\text{If } y_i = y_1 \text{ then } y'_i = 1 \text{; else } y'_i = y'_1 + 1
\end{equation}

\noindent For example if:

\texttt{x = {0,0,2,2,1,1}} and \texttt{y = {8,8,0,0,3,3}}

\noindent then:

\texttt{x' = {1,1,2,2,3,3}} and \texttt{y' = {1,1,2,2,3,3}}


\noindent The following proofs consider that the \textit{xyz} notation was applied.


\section{Distance metric proof}

The Character Difference ($CD$) metric is a distance $d: CD \times CD \rightarrow [0;\infty)$ with any character $x$, $y$ and $z$ (using the \textit{xyz} notation) $\in CD$ if an only if the followings are satisfied:

\begin{equation}
    \label{Positivity}
    CD_{(x,y)} \geq 0
\end{equation}
\begin{equation}
    \label{Identity}
    CD_{(x,y)} = 0 \Leftrightarrow x = y
\end{equation}
\begin{equation}
    \label{Equality}
    CD_{x,y} = CD_{y,x}
\end{equation}
\begin{equation}
    \label{Subadditivity}
    CD_{(x,y)} \leq CD_{(x,z)} + CD_{(z,y)}
\end{equation}

\noindent Furthermore, $CD$ is a evolutionary distance metric if it satisfies the four point condition:

\begin{equation}
    \label{fourpoint}
    CD_{(i,j)} + CD_{(m,n)} \leq \text{max}{(CD_{(i,m)} + CD_{(j,n)} ; CD_{(j,m)} + CD_{(i,n)} )}
\end{equation}

\newpage

\subsection{Character difference positivity}

For any $n \in [1;\infty)$ and $x_{n}$ and $y_{n} \in (0,1)$ because the difference between character states is considered as Fitch-like (unordered; i.e. the difference between two different characters states is equal to $1$ and the difference between two identical character states is equal to $0$):

\begin{equation}
    \label{positive_demo}
    \begin{gathered}
    0 \leq  \abs{x_{n}-y_{n}} \leq 1\\
    0 \leq  \sum_{i=1}^{n}\abs{x_{i}-y_{i}} \leq n\\
    0 \leq  \frac{\sum_{i}^{n}\abs{x_{i}-y_{i}}}{n} \leq 1\\
    0 \leq  \abs*{\frac{\sum_{i}^{n}\abs{x_{i}-y_{i}}}{n}-\frac{1}{2}} \leq \frac{1}{2}\\
    0 \leq  2\abs*{\frac{\sum_{i}^{n}\abs{x_{i}-y_{i}}}{n}-\frac{1}{2}} \leq 1\\
    0 \leq  1 - 2\left(\abs*{\frac{\sum_{i}^{n}\abs{x_{i}-y_{i}}}{n}-\frac{1}{2}}\right) \leq 1\\
    \end{gathered}
\end{equation}

Thus equation \ref{Positivity} is true.

\newpage

\subsection{Character difference identity}

For any $n \in [1;\infty)$ and $x_{n} = y_{n}$ then:
%TG: Is that valid for a demonstration?
\begin{equation}
    \label{identitydemo}
    \begin{aligned}
    \abs{x_{n}-y_{n}} &= 0\\
    \sum_{i}^{n}\abs{x_{i}-y_{i}} &= 0\\
    \frac{\sum_{i}^{n}\abs{x_{i}-y_{i}}}{n} &= 0\\
    \abs*{\frac{\sum_{i}^{n}\abs{x_{i}-y_{i}}}{n}-\frac{1}{2}} &= \frac{1}{2}\\
    2\abs*{\frac{\sum_{i}^{n}\abs{x_{i}-y_{i}}}{n}-\frac{1}{2}} &= 1\\
    1 - 2\left(\abs*{\frac{\sum_{i}^{n}\abs{x_{i}-y_{i}}}{n}-\frac{1}{2}}\right) &= 0\\
    \end{aligned}
\end{equation}

Thus equation \ref{Identity} is true.

\subsection{Character difference equality}

For any $n \in [1;\infty)$ and $x_{n}$ and $y_{n} \in (0,1)$, similarly to equation \ref{positive_demo}:

\begin{equation}
    \label{equality_demo}
    \begin{aligned}
    \abs{x_{n}-y_{n}} &= \abs{y_{n}-x_{n}}\\
    \sum_{i}^{n}\abs{x_{i}-y_{i}} &= \sum_{i}^{n}\abs{y_{i}-x_{i}}\\
    \frac{\sum_{i}^{n}\abs{x_{i}-y_{i}}}{n} &= \frac{\sum_{i}^{n}\abs{y_{i}-x_{i}}}{n}\\
    \abs*{\frac{\sum_{i}^{n}\abs{x_{i}-y_{i}}}{n}-\frac{1}{2}} &= \abs*{\frac{\sum_{i}^{n}\abs{y_{i}-x_{i}}}{n}-\frac{1}{2}}\\
    2\abs*{\frac{\sum_{i}^{n}\abs{x_{i}-y_{i}}}{n}-\frac{1}{2}} &= 2\abs*{\frac{\sum_{i}^{n}\abs{y_{i}-x_{i}}}{n}-\frac{1}{2}}\\
    1 - 2\left(\abs*{\frac{\sum_{i}^{n}\abs{x_{i}-y_{i}}}{n}-\frac{1}{2}}\right) &= 1 - 2\left(\abs*{\frac{\sum_{i}^{n}\abs{y_{i}-x_{i}}}{n}-\frac{1}{2}}\right)\\
    \end{aligned}
\end{equation}

Thus equation \ref{Equality} is true.

\subsection{Character difference subadditivity}

For any $n, j, b \in [1;\infty)$ and $x_{n}$, $y_{j}$, $z_{b}$ $\in (0,1)$:

\begin{equation}
    \label{subadditivity_demo}
    \begin{aligned}
    CD_{x,y} &\leq CD_{x,z} + CD_{z,y}\\
    \text{if:}\\
     1 - 2\left(\abs*{\frac{\sum_{i}^{n}\abs{x_{i}-y_{i}}}{n}-\frac{1}{2}}\right)
     &\leq 
     1 - 2\left(\abs*{\frac{\sum_{k}^{j}\abs{x_{k}-z_{k}}}{j}-\frac{1}{2}}\right) 
     +
     1 - 2\left(\abs*{\frac{\sum_{a}^{b}\abs{z_{a}-y_{a}}}{b}-\frac{1}{2}}\right) \\
     \text{then, the opposite is:}\\
     \underbrace{\abs*{\frac{\sum_{i}^{n}\abs{x_{i}-y_{i}}}{n}-\frac{1}{2}} }_{\in[0,1/2]}
     &\geq 
     \underbrace{\abs*{\frac{\sum_{k}^{j}\abs{x_{k}-z_{k}}}{j}-\frac{1}{2}} }_{\in[0,1/2]}
     +
     \underbrace{\abs*{\frac{\sum_{a}^{b}\abs{z_{a}-y_{a}}}{b}-\frac{1}{2}} }_{\in[0,1/2]} - 2\\
     \text{thus:}\\
     \in[0,\frac{1}{2}] &\geq \in[0,\frac{1}{2}] + \in[0,\frac{1}{2}] -2 \\
     \in[0,\frac{1}{2}] &\geq \in[-2,-1]\\
    \end{aligned}
\end{equation}

Thus equation \ref{Subadditivity} is true.


\newpage

\subsection{Four point condition}

For any $n \in [1;\infty)$ and any four distinct characters $i$, $j$, $n$ and $m$, and if equations \ref{Positivity} and \ref{Subadditivity} are true, the pairwise distances $CD_{(i,j)}$, $CD_{(m,n)}$, $CD_{(i,m)}$, $CD_{(j,n)}$, $CD_{(j,m)}$ and $CD_{(i,n)}$ can be represented as a combination of any five values $a$, $b$, $c$, $d$ and $e$ $\geq 0$:

\begin{figure}[!htbp]
\centering
   \includegraphics[width=0.5\textwidth]{Four_point.pdf}
\end{figure}

Then:

\begin{equation}
    \begin{aligned}
    CD_{(i,j)} = a + b \\
    CD_{(m,n)} = d + e \\
    CD_{(i,m)} = a + c + d \\
    CD_{(j,n)} = b + c + e \\
    CD_{(j,m)} = b + c + d \\
    CD_{(i,n)} = a + c + e \\
    \end{aligned}
\end{equation}

Thus:

\begin{equation}
    \begin{gathered}
    CD_{(i,j)} + CD_{(m,n)} \leq \text{max}{(CD_{(i,m)} + CD_{(j,n)} ; CD_{(j,m)} + CD_{(i,n)} )} \\ 
    a + b + d + e \leq \text{max} (a+c+d+b+c+e; b+c+d+a+c+e) \\
    a + b + d + e \leq a+b+2c+e+d \\
    2c \geq 0 \\
    \end{gathered}
\end{equation}

Thus equation \ref{fourpoint} is true.

\bibliographystyle{sysbio}
\bibliography{References}

\end{document}



