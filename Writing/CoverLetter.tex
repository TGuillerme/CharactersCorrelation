\documentclass[11pt]{letter}
\usepackage[a4paper,left=2.5cm, right=2.5cm, top=1cm, bottom=1cm]{geometry}
\usepackage{hyperref}
\usepackage[osf]{mathpazo}
\signature{Thomas Guillerme \\ Abigail Pastore \\ Martin Brazeau}
\address{The University of Queensland \\School of Biological Sciences \\St Lucia QLD 4067, Australia \\guillert@tcd.ie}
\longindentation=0pt
\begin{document}

\begin{letter}{}
\opening{Dear Editors,}

The assumption made in tree inference that phylogenetic data is independent is probably mostly erroneous.
This assumption about data correlation was stated since the advent of Cladistics in the 1950s but is nowadays mainly ignored in phylogenetic endeavour when using molecular data (where it can be neglected due to the huge amount of available data).
In morphological datasets, however, where the amount of characters are much smaller, correlation can lead to significant biases especially by favouring some topologies due to a specific choice of characters (whether this is done consciously by researchers or not).
With the ``resurgence'' of the use of morphological data in phylogenetic analysis (namely with the Total Evidence tip-dating method, Ronquist \textit{et al.}, 2012, Sys. Biol), it is thus important to acknowledge limitations and caveats that come with the use of such data, namely how correlation patterns in morphological datasets can effect tree inference.

In our research article, entitled ``Influence of different modes of morphological characters correlation on tree inference'', we propose the first - to our knowledge - testing of the effect of characters correlation on tree topology based on a thorough simulation framework ($>$ 100 CPU years worth).
Using a novel distance metric to measure direct phylogenetic correlation between characters, we simulated discrete morphological datasets that we then modified to increase or decrease the overall characters correlation within the datasets.
We then measured the resulting topological differences (from both maximum parsimony and Bayesian trees) to assess the effect of characters correlation in the light of our known simulated parameters.

Our results suggest that datasets that have higher levels of correlation between characters negatively impact our ability to recover the correct tree topology.
This is to be contrasted with datasets that have low levels of correlation (i.e. that are more homoplasic) which can still recover the correct topology, depending on the amount of data and the inference method used.
These results, although non-surprising to researchers in the field, are the first - to our knowledge - to actually quantify the effect of characters correlation and provides practical data and reusable code allowing researcher to effectively measure the limitations of their empirical datasets.
Furthermore, we suggest that using our proposed new metric, one could improve bootstrap procedures in phylogenetic analysis by ``biasing'' the bootstrap to favour characters with high or low correlation, depending on the purpose of the phylogenetic search.
Finally, the code and data for this whole analysis is entirely repeatable through \url{https://github.com/TGuillerme/CharactersCorrelation/}.

We look forward to hearing from you soon,

\closing{Yours sincerely,}


\end{letter}
\end{document}


%Editors:
%Sam Price, Daniele Silvestro

%Reviewers
%Graeme Lloyd, Dave Bapst, Emma Sherratt 
