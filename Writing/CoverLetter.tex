\documentclass[11pt]{letter}
\usepackage[a4paper,left=2.5cm, right=2.5cm, top=1cm, bottom=1cm]{geometry}
\usepackage[osf]{mathpazo}
\signature{Thomas Guillerme \\ Martin Brazeau}
\address{The University of Queensland \\School of Biological Sciences \\St Lucia QLD 4067, Australia \\guillert@tcd.ie}
\longindentation=0pt
\begin{document}

\begin{letter}{}
\opening{Dear Editors,}

Despite being at the base of almost every endeavour in modern Biology, the assumption made in tree inference that phylogenetic data is independent is probably mostly erroneous.
In fact, since the advent of Cladistics in the 1950s, evolutionary biologists have made the assumption that characters used to build a phylogeny are independent despite the fact that we are looking at their variance-covariance to build phylogenetic hypothesis.
This effect is somewhat negligible in molecular phylogenetics because the huge amount of available data is thought to temper down the correlation effect.
However, in smaller morphological datasets, this can lead to pretty big biases and can easily favour on topology from another simply with a ``judicious'' choice of characters.

Our research article, entitled ``Influence of different modes of morphological character correlation on tree inference'', is - to our knowledge - the first to thoroughly test the effect of character correlation on tree topology using a thorough simulation framework ($>$ 100 CPU years worth).
Using a novel metric to measure the amount of phylogenetic correlation in a matrix, we tested different scenario that either increase or decrease the amount of correlation of characters generated from a known ``true'' tree and compare how these levels of correlation affect our ability to recover this topology using both Maximum Parsimony and Bayesian inference.

Our results suggests that strong levels of correlation can negatively impact our ability to recover the correct tree topology and more so in Maximum Parsimony than in Baysian.
However, it is encouraging to see that low levels of correlation do not especially impact the topology, at least not to the extant where researchers should specifically collect only uncorrelated characters.
Finally, we show that, for smaller datasets, Maximum Parsimony and Bayesian inference are both pretty efficient to recover the correct topology, regardless of the amount of correlation

We look forward to hearing from you soon,

\closing{Yours sincerely,}


\end{letter}
\end{document}


%Editors:
%Sam Price, Daniele Silvestro

%Reviewers
%Graeme Lloyd, Dave Bapst, Emma Sherratt 
